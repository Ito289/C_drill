\documentclass[a4j]{jarticle}
\usepackage{multicol}
\raggedbottom
\addtolength{\textwidth}{4cm}
\addtolength{\textheight}{3.5cm}
\addtolength{\topmargin}{-2.5cm}
\addtolength{\evensidemargin}{-2cm}
\addtolength{\oddsidemargin}{-2cm}
\setlength{\columnseprule}{.5pt}
\setlength{\columnsep}{4zw}

\makeatletter
\def\verbatim@font{\small\bfseries\ttfamily}
\makeatother


%
% 1か2を指定する。
%
\def\anss{1} % 1 : Question mode 学生に解かせるとき
             % 2 : Answer mode   回答例



\ifnum \anss=1
\def\an#1{\phantom{#1}} % Question mode
\else
\def\an#1{{#1}}  % Answer mode
\fi



\def\ans#1#2{
\ifnum \anss=1
#1
\else
#2
\fi
}



\def\ds{\displaystyle}

\begin{document}
\thispagestyle{empty}

\begin{multicols*}{2}%


\def\subst#1#2{$\ds #1$
 \ $\longrightarrow$\ 
 \underline{\hbox to 5cm{\ttfamily #2}}}



\noindent
\begin{tabular}[t]{|c|cccccccc|}\hline
氏  名 & & & & & & & & \\ \hline
\end{tabular}\\
\begin{tabular}[t]{|c|c|c|c|c|c|c|c|c|c|}\hline
学籍番号 & & & & & & & & \\ \hline
\end{tabular}\\
学籍番号の\underline{数字の}右から一番目が{\bfseries 奇数の人は左側}の問題を
解いて下さい。
\vspace{-5ex}



\subsection*{問1}


1. 文字型の変数{\ttfamily x}と実数型の変数{\ttfamily c}
の変数を宣言し、それぞれ{\ttfamily 'j'と1.2}を代入して、
それらの値を{\ttfamily printf}文で表示するプログラムを書け。


\ifnum \anss=1
\begin{verbatim}








\end{verbatim}
\else
\begin{verbatim}
例:
char x = 'j';
float c = 1.2;
printf("%c %f", x, c);






\end{verbatim}
\fi



2. {\ttfamily 2*5*11*(6/7)*10*4*9}\\
\hfill の評価結果は\ans{____}{  0  }である。



3. \verb@y*8-5*θ@はC言語の式として\\
\hfill[\ans{正しい・間違っている}{正しい・\underline{間違っている}}]。

4. \verb@xy-4*YouAre_Japanese@はC言語の式として\\
\hfill[\ans{正しい・間違っている}{\underline{正しい}・間違っている}]。




\subsection*{問2}

次の文章を表示したい。
\begin{verbatim}
How are
you, John?
\end{verbatim}

以下のプログラム中の下線部を埋めよ。

\ifnum \anss=1
\begin{verbatim}
printf(_______________________);
\end{verbatim}
\else
\begin{verbatim}
printf("How are\nyou, John?");
\end{verbatim}
\fi









\subsection*{問4}

何が表示されるか。
\begin{verbatim}
int r = 1, a[5] = {5, 2, 6, 9, 8};
printf("%d ", a[r]);
r = r + 2;
printf("%d ", a[r]);
r = r + 2;
printf("\n");
\end{verbatim}

\ans{\mbox{}}{{\ttfamily 2 9}}


%
%\subsection*{問5}
%
%当てはまるものにチェックを付けよ。
%
%\begin{itemize}
% \itemsep=-3pt
% \item 配列は: □よく分からない。□理解した。
% \item 実数型と整数型の違いは:\\ □よく分からない。□理解した。
% \item {\ttfamily printf}関数の使いかたは:\\ □よく分からない。□理解した。
%\end{itemize}
%



\vspace*{3cm}

\vfill

\mbox{}

%{\vbox{\vspace{1cm}}}




\noindent
\begin{tabular}[t]{|c|cccccccc|}\hline
氏  名 & & & & & & & & \\ \hline
\end{tabular}\\
\begin{tabular}[t]{|c|c|c|c|c|c|c|c|c|c|}\hline
学籍番号 & & & & & & & & \\ \hline
\end{tabular}\\
学籍番号の\underline{数字の}右から一番目が{\bfseries 偶数の人は右側}の問題を
解いて下さい。
\vspace{-5ex}





\subsection*{問1}


1. 整数型の変数{\ttfamily y}と実数型の変数{\ttfamily z}
の変数を宣言し、それぞれ{\ttfamily 2と22.4}を代入して、
それらの値を{\ttfamily printf}文で表示するプログラムを書け。

\ifnum \anss=1
\begin{verbatim}








\end{verbatim}
\else
\begin{verbatim}
例:
int y = 2;
double z = 22.4;
printf("%d %f", y, z);






\end{verbatim}
\fi



2. {\ttfamily 3/9*2*5*11*4*9}の\\
\hfill 評価結果は\ans{____}{  0  }である。



3. \verb@3*π+2*x@はC言語の式として\\
\hfill[\ans{正しい・間違っている}{正しい・\underline{間違っている}}]。


4. \verb@ax+2*This_Is_A_Test@はC言語の式として\\
\hfill[\ans{正しい・間違っている}{\underline{正しい}・間違っている}]。





\subsection*{問2}

次の文章を表示したい。
\begin{verbatim}
Today is
very fine.
\end{verbatim}

以下のプログラム中の下線部を埋めよ。

\ifnum \anss=1
\begin{verbatim}
printf(_______________________);
\end{verbatim}
\else
\begin{verbatim}
printf("Today is\nvery fine.");
\end{verbatim}
\fi




\subsection*{問4}

何が表示されるか。
\begin{verbatim}
int w = 3, a[4] = {0, 9, 2, 8};
printf("%d ", a[w]);
w = w - 2;
printf("%d ", a[w]);
w = w - 2;
printf("\n");
\end{verbatim}

\ans{\mbox{}}{{\ttfamily 8 9}}


%
%\subsection*{問5}
%
%当てはまるものにチェックを付けよ。
%
%\begin{itemize}
% \itemsep=-3pt
% \item 配列は: □理解した。□よく分からない。
% \item 実数型と整数型の違いは:\\ □理解した。□よく分からない。
% \item {\ttfamily printf}関数の使いかたは:\\ □理解した。□よく分からない。
%\end{itemize}
%




%\vfill




\end{multicols*}


\end{document}
