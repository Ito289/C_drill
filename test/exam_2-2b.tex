\documentclass[a4j]{jarticle}
\usepackage{multicol}
\raggedbottom
\addtolength{\textwidth}{4cm}
\addtolength{\textheight}{3.5cm}
\addtolength{\topmargin}{-2.5cm}
\addtolength{\evensidemargin}{-2cm}
\addtolength{\oddsidemargin}{-2cm}
\setlength{\columnseprule}{.5pt}
\setlength{\columnsep}{4zw}

\makeatletter
\def\verbatim@font{\small\bfseries\ttfamily}
\makeatother



%
% 1か2を指定する。
%
\def\anss{1} % 1 : Question mode 学生に解かせるとき
             % 2 : Answer mode   回答例



\ifnum \anss=1
\def\an#1{\phantom{#1}} % Question mode
\else
\def\an#1{{#1}}  % Answer mode
\fi



\def\ans#1#2{
\ifnum \anss=1
#1
\else
#2
\fi
}





\ifnum \anss=1
\def\an#1{\phantom{#1}}
\else
\def\an#1{{#1}}
\fi



\def\ds{\displaystyle}

\begin{document}
\thispagestyle{empty}

\begin{multicols*}{2}%


\def\subst#1#2{$\ds #1$
 \ $\longrightarrow$\ 
 \underline{\hbox to 5cm{\ttfamily #2}}}



\noindent
\begin{tabular}[t]{|c|cccccccc|}\hline
氏  名 & & & & & & & & \\ \hline
\end{tabular}\\
\begin{tabular}[t]{|c|c|c|c|c|c|c|c|c|c|}\hline
学籍番号 & & & & & & & & \\ \hline
\end{tabular}\\
学籍番号の\underline{数字の}右から一番目が{\bfseries 奇数の人は左側}の問題を
解いて下さい。
\vspace{-5ex}


\subsection*{問1}


1. float型はdouble型より広い範囲の実数を
扱うことが[できない・できる]。

\ifnum \anss=1
\mbox{}
\else
できない
\fi


2. {\ttfamily 2/5*10}の評価結果は  \ifnum \anss=1\else 0 \fi  である。



\subsection*{問2}

何が表示されるか。
\begin{verbatim}
char what = 'm';
printf("%c%c%c\n", what, what + 1, what + 2);
\end{verbatim}

\ifnum \anss=1\else {\ttfamily mno} \fi


\subsection*{問3}


C言語の式として正しければ○を、間違っていれば×を書け。\\
\verb@5-5*y+x^2       @\underline{\an{  ×  }}\\
\verb@2*LastState-5   @\underline{\an{  ○  }}\\
\verb@Σ*x+2          @\underline{\an{  ×  }}\\




\subsection*{問4}

何が表示されるか。
\begin{verbatim}
int r = -4;
printf("%d ", r);
r = r + 2;
printf("%d ", r);
r = r + 2;
printf("%d ", r);
r = r + 2;
printf("\n");
\end{verbatim}
\an{\ttfamily -4 -2 0}



\subsection*{問5}


実数型の変数{\ttfamily x}と
整数型の変数{\ttfamily c}を宣言し、
それぞれ{\ttfamily 1.2}と{\ttfamily 33}で初期化して、
{\ttfamily printf}文によって二つの変数の値を
表示するためのプログラムを書け。

\ifnum \anss=1
\begin{verbatim}






\end{verbatim}
\else
\begin{verbatim}
例:
float x = 1.2;       doubleでも可
int c = 33;          (un)singed/long/shortでも可
printf("%f %d", x, c);


\end{verbatim}
\fi


%当てはまるものにチェックを付けよ。
%
%\begin{itemize}
% \itemsep=-3pt
% \item 配列は: □よく分からない。□理解した。
% \item 実数型と整数型の違いは:\\ □よく分からない。□理解した。
% \item {\ttfamily printf}関数の使いかたは:\\ □よく分からない。□理解した。
%\end{itemize}






\vfill

%\mbox{}

%{\vbox{\vspace{1cm}}}




\noindent
\begin{tabular}[t]{|c|cccccccc|}\hline
氏  名 & & & & & & & & \\ \hline
\end{tabular}\\
\begin{tabular}[t]{|c|c|c|c|c|c|c|c|c|c|}\hline
学籍番号 & & & & & & & & \\ \hline
\end{tabular}\\
学籍番号の\underline{数字の}右から一番目が{\bfseries 偶数の人は右側}の問題を
解いて下さい。
\vspace{-5ex}





\subsection*{問1}

1. unsigned int型で{\ttfamily -4}は[表せる・表せない]。\\

\an{表せない}


2. {\ttfamily 10*2/5}の評価結果は  \an{4}  である。



\subsection*{問2}

何が表示されるか。
\begin{verbatim}
char hey = 'u';
printf("%c%c%c\n", hey - 1, hey, hey + 1);
\end{verbatim}
\an{{\ttfamily tuv}}



\subsection*{問3}


C言語の式として正しければ○を、間違っていれば×を書け。\\
\verb@2*sigma\c           @\underline{\an{  ×  }}\\
\verb@OutputResult*2-x     @\underline{\an{  ○  }}\\
\verb@33+5*x*π            @\underline{\an{  ×  }}\\



\subsection*{問4}

何が表示されるか。
\begin{verbatim}
int w = 9;
printf("%d ", w);
w = w - 2;
printf("%d ", w);
w = w - 2;
printf("%d ", w);
w = w - 2;
printf("\n");
\end{verbatim}
\an{{\ttfamily 9 7 5}}



\subsection*{問5}



実数型の変数{\ttfamily z}と
文字型の変数{\ttfamily d}を宣言し、
それぞれ{\ttfamily 0.23}と{\ttfamily 'T'}で初期化して、
{\ttfamily printf}文によって二つの変数の値を
表示するためのプログラムを書け。

\ifnum \anss=1
\begin{verbatim}






\end{verbatim}
\else
\begin{verbatim}
例:
float z = 0.23;    doubleでも可
char d = 'T';      (un)singedでも可
printf("%f %c", z, d);  "%f %d"でも可


\end{verbatim}
\fi
%
%
%当てはまるものにチェックを付けよ。
%
%\begin{itemize}
% \itemsep=-3pt
% \item 配列は: □理解した。□よく分からない。
% \item 実数型と整数型の違いは:\\ □理解した。□よく分からない。
% \item {\ttfamily printf}関数の使いかたは:\\ □理解した。□よく分からない。
%\end{itemize}
%




%\vfill




\end{multicols*}


\end{document}
