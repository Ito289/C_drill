\documentclass[a4j]{jarticle}
\usepackage{multicol}
\usepackage{fancybox}

\raggedbottom
\addtolength{\textwidth}{4cm}
\addtolength{\textheight}{4cm}
\addtolength{\topmargin}{-2.5cm}
%\addtolength{\topmargin}{-1.5cm}
\addtolength{\evensidemargin}{-2cm}
\addtolength{\oddsidemargin}{-2cm}
\setlength{\columnseprule}{.5pt}
\setlength{\columnsep}{4zw}

\makeatletter
\def\verbatim@font{%\small
\bfseries\ttfamily}
\makeatother



%
% 1か2を指定する。
%
\def\anss{1} % 1 : Question mode 学生に解かせるとき
             % 2 : Answer mode   回答例



\ifnum \anss=1
\def\an#1{\phantom{#1}} % Question mode
\else
\def\an#1{{#1}}  % Answer mode
\fi


\def\ans#1#2{
\ifnum \anss=1
#1
\else
#2
\fi
}

\newcounter{toi}
\setcounter{toi}{1}
\def\toi{%
%\subsubsection*{問\thetoi}
\bigskip\bigskip\noindent
\raisebox{-1.2ex}{
\shadowbox{\bfseries\large 問題\thetoi}
}
%\addcontentsline{toc}{subsubsection}{問\thetoi}
\addtocounter{toi}{1}
%\nopagebreak[4]\bigskip\nopagebreak[4]
}


\setlength{\topsep}{0pt}



\def\ds{\displaystyle}

\begin{document}
\thispagestyle{empty}
\pagestyle{empty}

\begin{multicols*}{2}%


\def\subst#1#2{$\ds #1$
 \ $\longrightarrow$\ 
 \underline{\hbox to 5cm{\ttfamily #2}}}



\noindent
\begin{tabular}[t]{|c|cccccccc|}\hline
氏  名 & & & & & & & & \\ \hline
\end{tabular}\\
\begin{tabular}[t]{|c|c|c|c|c|c|c|c|c|c|}\hline
学籍番号 & & & & & & & & \\ \hline
\end{tabular}\\



\toi
次のC言語の式を評価せよ。
\begin{itemize}
 \item \verb@1+5*3/6@  解答:\ans{_____}{3}
 \item \verb@(-1<4)||(7>8)@  解答:\ans{_____}{0}
 \item \verb@4<5>3@  解答:\ans{_____}{0}
\end{itemize}


\toi
次の数学の式をC言語の式に書き直せ。
\begin{itemize}
 \item $\ds\frac{2-1}{1-\ds\frac{4}{3+1}}-3$\\[2ex]
       解答:\ans{}{{\ttfamily\small
       (2-1)/(1-4/(3+1))-3
       }}\\
 \item $\cos\frac{2x^2+2xy+1}{\sqrt{2}}$\\[2ex]
       解答:\ans{}{{\ttfamily\small
       cos((2*x*x+2*x*y+1)/sqrt(2))}}
\end{itemize}


\toi
次の数式を代入文で書こうとしたが、間違っている。
訂正せよ。
\begin{itemize}
 \itemsep=5ex
 \item $x=\ds\frac{a+b}{-\frac{c+2}{d}-e}$
      $\quad\longrightarrow\quad$
      \ans{\ttfamily x=(a+b)/[-\{c+2\}d-e];}
      {\ttfamily x=(a+b)/(-(c+2)/d-e);}
\end{itemize}



%
%
%\toi
%15と12の最大公約数と最小公倍数を求めよ。\\
%最大公約数:\ans{_____}{5}, \quad 最小公倍数:\ans{_____}{60}
%


\toi
何が表示されるか。
\begin{verbatim}
    int a, b, c;
    a = 3;
    b = 2;
    b = b - 5;
    a++;
    b += a;
    c = a + b; 
    printf("%d %d %d\n", a, c, b);

\end{verbatim}
\noindent
解答:\ans{____________}{4 5 1}





\toi
実数型の変数{\ttfamily ax}と、
符号付き整数型の変数{\ttfamily by}
の変数宣言を書け。
{\footnotesize (詳しく指定しない。
当てはまるものを書け)}

\ifnum \anss=1
\begin{verbatim}



\end{verbatim}
\else
\begin{verbatim}
    float ax;  or double ax;

    int by; or signed int/short/long by;
\end{verbatim}
\fi






\toi
何が表示されるか。
\begin{verbatim}
    float a=4.8, b=2.4, x;
    int m=5, n=2;
    x = a / b + m / n;
    if(x <= 4){
      printf("4以下\n");
    }else{
      printf("4より大\n");
    }
\end{verbatim}
\noindent
解答:\ans{}{4以下}




\toi
以下の質問に解答せよ。\\
・{\bfseries\small
問題集をやったことが実際のプログラミングの時に役に立ったことが}
[{\scriptsize ある/ない}]\\
{\scriptsize 自由記述欄:}
_____________________\\
・{\bfseries\small
 問題集をやった期間{\scriptsize (5週間)}は}
[{\scriptsize 長かった/短かかった}]\\
{\scriptsize 自由記述欄:}
_____________________\\
・{\bfseries\small 問題の量は}
[{\scriptsize 多い/少ない}]\\
{\scriptsize 自由記述欄:}
_____________________\\
・{\bfseries\small 問題の出題分野は}
[{\scriptsize 広い・十分/狭い・足りない}]\\
{\scriptsize 自由記述欄:}
_____________________\\
・{\bfseries\small 問題集と課題について感想を述べよ。}\\
{\scriptsize 自由記述欄:}\\
%_____________________



%
%
%\toi
%以下の質問に解答せよ。\\
%・{\bfseries\small
%問題集でやったことが課題を解く時に役に立ったことが}
%[{\scriptsize ある/ない}]\\
%{\scriptsize 自由記述欄:}
%_____________________\\
%・{\bfseries\small
% 問題集をやった期間
%{\scriptsize (4週間)}は}
%[{\scriptsize 長かった/短かかった}]\\
%{\scriptsize 自由記述欄:}
%_____________________\\
%・{\bfseries\small 問題の量は}
%[{\scriptsize 多い/少ない}]\\
%{\scriptsize 自由記述欄:}
%_____________________\\
%・{\bfseries\small 問題の出題分野は}
%[{\scriptsize 広い・十分/狭い・足りない}]\\
%{\scriptsize 自由記述欄:}
%_____________________\\
%・{\bfseries 問題を追加・削除した方がいい分野は?}\\
%{\scriptsize 自由記述欄:}
%_____________________\\
%
%









%
%\toi\\
%以下の質問に答えよ。
%{\footnotesize (分野は問題4から選んでもよい)}\\
%・難しかったのはどの分野ですか?{\footnotesize (複数回答可)}\\
%______________________\\
%・理解できなかったのはどの分野ですか?{\footnotesize (複数回答可)}\\
%______________________\\
%・実習以前のプログラミング経験は?
%{\footnotesize (何年程度か、言語は)}\\
%______________________\\
%・今後、他教科で出題されたレポート課題において、
%自分でプログラムを作れると[思う・思わない]、また
%作ってやってみようと[思う・思わない]。
%



\toi
for文を使って、配列{\ttfamily Name[912]}
の全ての要素の和を計算して
表示したい。以下のプログラムの
下線部を埋めよ。ただし{\ttfamily Name}はすでに
宣言され初期化されているとする。

\ifnum \anss=1
\begin{verbatim}
int h, number=0;
for( ______ ; _____ ; _____ ){
   number += Name[h];
}
printf("%d\n", number);
\end{verbatim}
\else
\begin{verbatim}
int h, number=0;
for(h = 0; h < 912; h++){
   number += Name[h];
}
printf("%d\n", number);
\end{verbatim}
\fi







\toi
何が表示されるか。
\begin{verbatim}
int x, y, n = 0;
for(x = 0; x < 10; x++){
  for(y = 0; y < 10; y++){
  };
  n++;
}
printf("n is %d\n", n);

\end{verbatim}
解答:\ans{__________}{{\ttfamily n is 10}}



\pagebreak


%\def\nums{\hfill 1 2 3 4 5}
\def\nums{\hfill 1&2&3&4&5}

\toi
以下の各分野の「現在の」理解度について、
当てはまる数字を選べ。
 {\bfseries\footnotesize 1.理解している/
2.だいたい分かっている/
3.まあなんとなく/
4.まだよく分かっていない/
5.さっぱり}
\begin{center}\small
% \begin{tabular}{|r|c|}\hline
 \begin{tabular}{|r||c|c|c|c|c|c|}\hline
 式の評価&\nums\\\hline
 評価順序&\nums\\\hline
 変数への代入&\nums\\\hline
 整数型と実数型の違い&\nums\\\hline
 {\ttfamily printf}関数&\nums\\\hline
 配列&\nums\\\hline
 {\ttfamily while}&\nums\\\hline
 {\ttfamily for}&\nums\\\hline
 二重ループ&\nums\\\hline
 無限ループ&\nums\\\hline
 剰余&\nums\\\hline
 {\ttfamily if}&\nums\\\hline
 {\ttfamily else}&\nums\\\hline
 {\ttfamily if}とループの組合せ&\nums\\\hline
 {\ttfamily \&\&}や{\ttfamily ||}&\nums\\\hline
 フローチャートを読む&\nums\\\hline
 フローチャートを描く&\nums\\\hline
 ポインタ&\nums\\\hline
 \end{tabular}
\end{center}






\toi
配列{\ttfamily b}の要素のうち、
ある変数{\ttfamily y}より大きいものを全て表示するプログラムを、
ループを使って書け。次がその一行目とする。
ただし{\ttfamily y}がどのような値であっても
通用するように書くこと。
\ifnum \anss=1
\begin{verbatim}
int j, b[5] = {2, 44, 99, 35, 41}, y = 38;














\end{verbatim}
\else
\begin{verbatim}
int j, b[5] = {2, 44, 99, 35, 41}, y = 38;
j = 0;
while(j < 5){
  if(b[j] > y){
  printf("%d ", b[j]);
  }
  j++;
}
または
int j, b[5] = {2, 44, 99, 35, 41}, y = 38;
for(j = 0; j < 5; j++){
  if(b[j] > y){
  printf("%d ", b[j]);
  }
}
\end{verbatim}
\fi





\toi
以下の代入文が上から順に実行されたとき、
各変数の値を書け。
\begin{verbatim}
int x = 1, *p, y = 5;
p = &y;
y = 3;
x = *p + 1;
\end{verbatim}
\noindent
解答\\
{\ttfamily  x}の値:\ans{___}{4} %
{\ttfamily *p}の値:\ans{___}{3} %
{\ttfamily  y}の値:\ans{___}{3}







\toi
次のプログラムに対応するフローチャートを描け。
\begin{verbatim}
int i, n = 0;
for(i = 12; i > -121; i -= 2){
  n += 3 * i - 1;
  printf("%d ", n);
}  
\end{verbatim}


%
%\vfill\vfill\vfill\vfill\vfill\vfill
%\toi
%問題集と課題について感想を述べよ。


\end{multicols*}


\end{document}
