\documentclass[a4j]{jarticle}
\usepackage{multicol}
\raggedbottom
\addtolength{\textwidth}{4cm}
\addtolength{\textheight}{3.5cm}
\addtolength{\topmargin}{-2.5cm}
\addtolength{\evensidemargin}{-2cm}
\addtolength{\oddsidemargin}{-2cm}
\setlength{\columnseprule}{.5pt}
\setlength{\columnsep}{4zw}

\makeatletter
\def\verbatim@font{\small\bfseries\ttfamily}
\makeatother


%
% 1か2を指定する。
%
\def\anss{1} % 1 : Question mode 学生に解かせるとき
             % 2 : Answer mode   回答例



\ifnum \anss=1
\def\an#1{\phantom{#1}} % Question mode
\else
\def\an#1{{#1}}  % Answer mode
\fi


\def\ds{\displaystyle}

\begin{document}
\thispagestyle{empty}

\begin{multicols*}{2}%


\def\subst#1#2{$\ds #1$
 \ $\longrightarrow$\ 
 \underline{\hbox to 5cm{\ttfamily #2}}}



\noindent
\begin{tabular}[t]{|c|cccccccc|}\hline
氏  名 & & & & & & & & \\ \hline
\end{tabular}\\
\begin{tabular}[t]{|c|c|c|c|c|c|c|c|c|c|}\hline
学籍番号 & & & & & & & & \\ \hline
\end{tabular}\\
学籍番号の\underline{数字の}右から一番目が{\bfseries 奇数の人は左側}の問題を
解いて下さい。
\vspace{-5ex}


\subsection*{問1}

次のC言語の式を評価せよ。
\begin{center}
 \begin{tabular}{|l|c|}\hline
 式 & 評価結果 \\\hline\hline
 \verb;20 / 5*2; & \an{8} \\\hline
 \verb;27/9!=5-2;& \an{0} \\\hline
 \verb;(5*2+4*2)/(2+1)*4;& \an{24}\\\hline
 \verb;((0<1)||(2>4))&&((6-3)/3);& \an{1} \\\hline
 \end{tabular}
\end{center}


\subsection*{問2}

次の数式をC言語の式に書き直せ。
ただし$e$は指数とする。

\begin{itemize}
% \itemsep=12ex
 \item $\ds 1+\frac{-\frac{3-2}{8}-11}{-55-1}-4$

      \quad \rule[-3ex]{0pt}{6ex}
       \an{{\ttfamily
       1+(-(3-2)/8-11)/(-55-1)-4
       }}

 \item $\frac{1}{2\pi|\sigma|}
       \ e^{-x^2}$

       \quad \rule[-3ex]{0pt}{6ex}
       \quad \an{{\ttfamily 1/(2*pi*fabs(sigma))*exp(-x*x)}}

       \quad \rule[-3ex]{0pt}{6ex}



 \item $\ds\frac{x^n-1}{(x-1)^2}$

      \quad \rule[-3ex]{0pt}{6ex}
       \an{{\ttfamily
       (pow(x,n)-1)/((x-1)*(x-1))
       }}

\end{itemize}





\subsection*{問3}

以下の代入文が上から順番に実行されるとき、全ての代入文の
実行が終わった時点での各変数の値を書け。

\begin{verbatim}
x = 3;
y = 2;
x = y + 6;
y = x / 2;
x = x + 4;
y = (x > 1) * 2 + z * 4 + (x < 10);
z = x / 6;
\end{verbatim}


{\ttfamily x}の値:\an{\quad 12}\\

{\ttfamily y}の値:\an{\quad 未定}\\

{\ttfamily z}の値:\an{\quad 2}\\





\vfill



%{\vbox{\vspace{1cm}}}




\noindent
\begin{tabular}[t]{|c|cccccccc|}\hline
氏  名 & & & & & & & & \\ \hline
\end{tabular}\\
\begin{tabular}[t]{|c|c|c|c|c|c|c|c|c|c|}\hline
学籍番号 & & & & & & & & \\ \hline
\end{tabular}\\
学籍番号の\underline{数字の}右から一番目が{\bfseries 偶数の人は右側}の問題を
解いて下さい。
\vspace{-5ex}





\subsection*{問1}

次のC言語の式を評価せよ。
\begin{center}
 \begin{tabular}{|l|c|}\hline
 式 & 評価結果 \\\hline\hline
 \verb;36 / 9/2;& \an{2} \\\hline
 \verb;256/2<=+120-5;& \an{0}  \\\hline
 \verb;!(8<8)*3;& \an{3} \\\hline
 \verb;((0<1)&&(2>4))||((6-3)/3);& \an{1}\\\hline
 \end{tabular}
\end{center}


\subsection*{問2}

次の数式をC言語の式に書き直せ。
ただし$e$は指数とする。

\begin{itemize}

 \item $\ds \frac{-4+11}{\frac{-5}{-3-33}+1}-2$

       \quad \rule[-3ex]{0pt}{6ex}
       \an{{\ttfamily
       (-4+11)/(-5/(-3-33)+1)-2
       }}

 \item $\sin\alpha\cos\beta+\cos\alpha\sin\beta$

       \quad \rule[-3ex]{0pt}{6ex}
       \an{{\ttfamily
       sin(alpha)*cos(beta)+cos(alpha)*sin(beta)
       }}
       \quad \rule[-3ex]{0pt}{6ex}

 \item $\ds\frac{a^p}{a^q}-a^{p-q}$

       \quad \rule[-3ex]{0pt}{6ex}
       \an{{\ttfamily
       pow(a,p)/pow(a,q)-pow(a,p-q)
       }}



\end{itemize}





\subsection*{問3}

以下の代入文が上から順番に実行されるとき、全ての代入文の
実行が終わった時点での各変数の値を書け。

\begin{verbatim}
x = (y = 3) + 1;
y = y + 2;
x = y + 7;
y = x / 2;
x = x + 4;
y = (x > 1) * 2 + z * 4 + (x < 10);
z = x * 2;
\end{verbatim}


{\ttfamily x}の値:\an{\quad 16}\\

{\ttfamily y}の値:\an{\quad 未定}\\

{\ttfamily z}の値:\an{\quad 32}\\






\vfill




\end{multicols*}


\end{document}
