\documentclass[a4j]{jarticle}
\usepackage{multicol}
\raggedbottom
\addtolength{\textwidth}{4cm}
\addtolength{\textheight}{3.5cm}
\addtolength{\topmargin}{-2.5cm}
\addtolength{\evensidemargin}{-2cm}
\addtolength{\oddsidemargin}{-2cm}
\setlength{\columnseprule}{.5pt}
\setlength{\columnsep}{4zw}

\makeatletter
\def\verbatim@font{\small\bfseries\ttfamily}
\makeatother



%
% 1か2を指定する。
%
\def\anss{1} % 1 : Question mode 学生に解かせるとき
             % 2 : Answer mode   回答例



\ifnum \anss=1
\def\an#1{\phantom{#1}} % Question mode
\else
\def\an#1{{#1}}  % Answer mode
\fi


\def\ans#1#2{
\ifnum \anss=1
#1
\else
#2
\fi
}





\def\ds{\displaystyle}

\begin{document}
\thispagestyle{empty}

\begin{multicols*}{2}%


\def\subst#1#2{$\ds #1$
 \ $\longrightarrow$\ 
 \underline{\hbox to 5cm{\ttfamily #2}}}



\noindent
\begin{tabular}[t]{|c|cccccccc|}\hline
氏  名 & & & & & & & & \\ \hline
\end{tabular}\\
\begin{tabular}[t]{|c|c|c|c|c|c|c|c|c|c|}\hline
学籍番号 & & & & & & & & \\ \hline
\end{tabular}\\
学籍番号の\underline{数字の}右から一番目が{\bfseries 奇数の人は左側}の問題を
解いて下さい。
\vspace{-5ex}






\subsection*{問1}

{\ttfamily while}文を用いたプログラムに書き直せ。
\begin{verbatim}
int r = -4;
printf("%d ", r);
r = r + 2;
printf("%d ", r);
r = r + 2;
printf("%d ", r);
r = r + 2;
printf("\n");
\end{verbatim}

\ifnum \anss=1
\begin{verbatim}






\end{verbatim}
\else
\begin{verbatim}
int r = -4;
while(r < 1){  または (r<2) または (r<=0)
  printf("%d ", r);
  r = r + 2;
}
printf("\n");
\end{verbatim}
\fi

\vspace*{7cm}




\subsection*{問2}

何が表示されるか。

\begin{verbatim}
  int i = 0, j;
  while(i < 3){
    j = 0;
    i = i + 1;
    while(j < 3){
      printf("%d ", i+j);
      j = j + 1;
    }
    printf("\n");
  }
\end{verbatim}


\ifnum \anss=1
\begin{verbatim}



\end{verbatim}
\else
\begin{verbatim}
1 2 3
2 3 4
3 4 5
\end{verbatim}
\fi





%
%何が表示されるか。
%\begin{verbatim}
%int i = 0, a[6] = {11, 56, 34, 77, 39, 90};
%while(i < 6){
%  if(a[i] < 50){
%     printf("** ");
%  }else{
%     printf("%d ", a[i]); 
%  }
%}
%\end{verbatim}
%\an{{\ttfamily ** 56 ** 77 ** 90}}
%
%\vspace*{5cm}


\vfill

\mbox{}

%{\vbox{\vspace{1cm}}}




\noindent
\begin{tabular}[t]{|c|cccccccc|}\hline
氏  名 & & & & & & & & \\ \hline
\end{tabular}\\
\begin{tabular}[t]{|c|c|c|c|c|c|c|c|c|c|}\hline
学籍番号 & & & & & & & & \\ \hline
\end{tabular}\\
学籍番号の\underline{数字の}右から一番目が{\bfseries 偶数の人は右側}の問題を
解いて下さい。
\vspace{-5ex}




\subsection*{問1}


{\ttfamily while}文を用いたプログラムに書き直せ。
\begin{verbatim}
int w = 9;
printf("%d ", w);
w = w - 2;
printf("%d ", w);
w = w - 2;
printf("%d ", w);
w = w - 2;
printf("\n");
\end{verbatim}


\ifnum \anss=1
\begin{verbatim}






\end{verbatim}
\else
\begin{verbatim}
int w = 9;
while(w > 4){    または (w>3) または (w>=5)
  printf("%d ", w);
  w = w - 2;
}
printf("\n");
\end{verbatim}
\fi
\vspace*{7cm}




\subsection*{問2}


何が表示されるか。



\begin{verbatim}
  int i = 0, j;
  while(i < 9){
    j = 0;
    while(j < 3){
      j = j + 1;
      printf("%d ", i+j);
      i = i + 1;
    }
    printf("\n");
  }
\end{verbatim}


\ifnum \anss=1
\begin{verbatim}



\end{verbatim}
\else
\begin{verbatim}
1 3 5
4 6 8
7 9 11
\end{verbatim}
\fi






%\begin{verbatim}
%int i = 0, a[6] = {11, 56, 34, 77, 39, 90};
%while(i < 6){
%  if(a[i] > 40){
%     printf("%d ", a[i]);
%  }else{
%     printf("** ");
%  }
%}
%\end{verbatim}
%\an{{\ttfamily ** 56 ** 77 ** 90}}
%





%\vfill




\end{multicols*}


\end{document}
