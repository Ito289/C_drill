\documentclass[a4j]{jarticle}
\usepackage{multicol}
\raggedbottom
\addtolength{\textwidth}{4cm}
\addtolength{\textheight}{3.5cm}
\addtolength{\topmargin}{-2.5cm}
\addtolength{\evensidemargin}{-2cm}
\addtolength{\oddsidemargin}{-2cm}
%\setlength{\columnseprule}{.5pt}
\setlength{\columnsep}{4zw}

\makeatletter
\def\verbatim@font{\small\bfseries\ttfamily}
\makeatother



%
% 1か2を指定する。
%
\def\anss{1} % 1 : Question mode 学生に解かせるとき
             % 2 : Answer mode   回答例



\ifnum \anss=1
\def\an#1{\phantom{#1}} % Question mode
\else
\def\an#1{{#1}}  % Answer mode
\fi


\def\ans#1#2{
\ifnum \anss=1
#1
\else
#2
\fi
}




\def\ds{\displaystyle}

\begin{document}
\thispagestyle{empty}

\begin{multicols*}{2}%


\def\subst#1#2{$\ds #1$
 \ $\longrightarrow$\ 
 \underline{\hbox to 5cm{\ttfamily #2}}}



\noindent
\begin{tabular}[t]{|c|cccccccc|}\hline
氏  名 & & & & & & & & \\ \hline
\end{tabular}\\
\begin{tabular}[t]{|c|c|c|c|c|c|c|c|c|c|}\hline
学籍番号 & & & & & & & & \\ \hline
\end{tabular}\\
学籍番号の\underline{数字の}右から一番目が{\bfseries 奇数の人は左側}の問題を
解いて下さい。
\vspace{-5ex}







\subsection*{問1}

次の虫食い算を解くプログラムを(二重ループと{\ttfamily if}
と剰余を用いて)書け。
\begin{verbatim}
 1□□
× 1□
────
□□□1
□□□
────
□11□
\end{verbatim}
つまり、$100\times10$から$199\times19$のかけ算{\ttfamily a*b}の
二重ループを考え、
その{\ttfamily aとb}の組が筆算の{\bfseries □}の条件を満たすかどうか(一桁目が1かど
うか、など)を
{\ttfamily if}文で判定し、
筆算の計算結果を{\ttfamily printf}で表示する。


\ifnum \anss=1
\begin{verbatim}

\end{verbatim}
\vspace{14cm}
\vfill
\mbox{}\\
\else
\begin{verbatim}
int a = 100, b;
while(a <= 199){
  b = 10;
  while(b <= 19){
    if( ((b-10)*a) % 10 == 1 &&
        1000 <= ((b-10)*a) && ((b-10)*a) < 10000 &&
        ((a*b)/10) % 100 == 11 &&
        1000 <= b*a && b*a < 10000
        ){

      printf(" %d\n", a);
      printf("x %d\n", b);
      printf("----\n");
      printf("%d\n", (b-10)*a);
      printf("%d\n",  a);
      printf("----\n");
      printf("%d\n",  a*b);
    }
    b = b + 1;
  }
  a = a + 1;
}


正解  まちがい
 183   101
x 17  x 11
----  ----
1281   101
183   101
----  ----
3111  1111
\end{verbatim}
\vspace{3cm}
\mbox{}
\fi








\noindent
\begin{tabular}[t]{|c|cccccccc|}\hline
氏  名 & & & & & & & & \\ \hline
\end{tabular}\\
\begin{tabular}[t]{|c|c|c|c|c|c|c|c|c|c|}\hline
学籍番号 & & & & & & & & \\ \hline
\end{tabular}\\
学籍番号の\underline{数字の}右から一番目が{\bfseries 偶数の人は右側}の問題を
解いて下さい。
\vspace{-5ex}









\subsection*{問1}

次の虫食い算を解くプログラムを(二重ループと{\ttfamily if}
と剰余を用いて)書け。
\begin{verbatim}
 2□□
× 2□
────
2□□2
□2□
────
□□□□
\end{verbatim}
つまり、$200\times20$から$299\times29$のかけ算{\ttfamily a*b}の
二重ループを考え、
その{\ttfamily aとb}の組が筆算の{\bfseries □}の条件を満たすかどうか(二桁目が2かど
うか、など)を
{\ttfamily if}文で判定し、
筆算の計算結果を{\ttfamily printf}で表示する。


\ifnum \anss=1
\begin{verbatim}

\end{verbatim}
\vspace{5cm}
\else
\begin{verbatim}
int a = 200, b;
while(a <= 299){
  b = 20;
  while(b <= 29){
    if( ((b-20)*a) % 10 == 2 &&
        2000 <= ((b-20)*a) && ((b-20)*a) < 3000 &&
        ((a*2)/10) % 10 == 2 &&
        1000 <= b*a && b*a < 10000
        ){

      printf(" %d\n", a);
      printf("x %d\n", b);
      printf("----\n");
      printf("%d\n", (b-20)*a);
      printf("%d\n",  a*2);
      printf("----\n");
      printf("%d\n\n",  a*b);
    }
    b = b + 1;
  }
  a = a + 1;
}


正解
 264
x 28
----
2112
528
----
7392
\end{verbatim}
\fi









\end{multicols*}


\end{document}
