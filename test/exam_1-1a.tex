\documentclass[a4j]{jarticle}
\usepackage{multicol}
\raggedbottom
\addtolength{\textwidth}{4cm}
\addtolength{\textheight}{3.5cm}
\addtolength{\topmargin}{-2.5cm}
\addtolength{\evensidemargin}{-2cm}
\addtolength{\oddsidemargin}{-2cm}
\setlength{\columnseprule}{.5pt}
\setlength{\columnsep}{4zw}

\makeatletter
\def\verbatim@font{\small\bfseries\ttfamily}
\makeatother


%
% 1か2を指定する。
%
\def\anss{1} % 1 : Question mode 学生に解かせるとき
             % 2 : Answer mode   回答例



\ifnum \anss=1
\def\an#1{\phantom{#1}} % Question mode
\else
\def\an#1{{#1}}  % Answer mode
\fi



\def\ds{\displaystyle}

\begin{document}
\thispagestyle{empty}

\begin{multicols*}{2}%


\def\subst#1#2{$\ds #1$
 \ $\longrightarrow$\ 
 \underline{\hbox to 5cm{\ttfamily #2}}}



\noindent
\begin{tabular}[t]{|c|cccccccc|}\hline
氏  名 & & & & & & & & \\ \hline
\end{tabular}\\
\begin{tabular}[t]{|c|c|c|c|c|c|c|c|c|c|}\hline
学籍番号 & & & & & & & & \\ \hline
\end{tabular}\\
学籍番号の\underline{数字の}右から一番目が{\bfseries 奇数の人は左側}の問題を
解いて下さい。
\vspace{-5ex}


\subsection*{問1}

次のC言語の式を評価せよ。
\begin{center}
 \begin{tabular}{|l|c|}\hline
 式 & 評価結果 \\\hline\hline
 \verb;20 / 4*2; & \an{10} \\\hline
 \verb;21/7!=4-1;& \an{0} \\\hline
 \verb;(7+8-5+17)/(4-1)*4;& \an{36}\\\hline
 \verb;((0>1)||(2<4))&&((4-2)/2);& \an{1} \\\hline
 \end{tabular}
\end{center}


\subsection*{問2}

次の数式をC言語の式に書き直せ。
ただし$e$は指数とする。

\begin{itemize}
% \itemsep=12ex
 \item $\ds1+\frac{1-\frac{3-2}{8}}{22-5}$

      \quad \rule[-3ex]{0pt}{6ex}
       \an{{\ttfamily 1+(1-(3-2)/8)/(22-5)  }}

 \item $\frac{1}{\sqrt{2\pi\sigma^2}}
      \ e^{-\frac{(x-\mu)^2}{2\sigma^2}}$

       \quad \rule[-3ex]{0pt}{6ex}
       \quad \an{{\ttfamily 1/sqrt(2*pi*sigma*sigma)*}}

       \quad \rule[-3ex]{0pt}{6ex}
       \an{{\ttfamily exp(-(x-mu)*(x-mu)/(2*sigma*sigma))}}


 \item $\left(\ds\frac{4-2x}{1+5y}\right)^n$

      \quad \rule[-3ex]{0pt}{6ex}
       \an{{\ttfamily
       pow( (4-2*x)/(1+5*y), n )
       }}

\end{itemize}





\subsection*{問3}

以下の代入文が上から順番に実行されるとき、全ての代入文の
実行が終わった時点での各変数の値を書け。

\begin{verbatim}
x = 1;
y = 1;
x = 3 + 5;
y = x / 2;
x = x + 4;
y = (x > 1) * 2 + z + (x < 10) * 4;
z = (x + y) * 2;
\end{verbatim}


{\ttfamily x}の値:\an{\quad 12}\\

{\ttfamily y}の値:\an{\quad 未定}\\

{\ttfamily z}の値:\an{\quad 未定}\\





\vfill



%{\vbox{\vspace{1cm}}}




\noindent
\begin{tabular}[t]{|c|cccccccc|}\hline
氏  名 & & & & & & & & \\ \hline
\end{tabular}\\
\begin{tabular}[t]{|c|c|c|c|c|c|c|c|c|c|}\hline
学籍番号 & & & & & & & & \\ \hline
\end{tabular}\\
学籍番号の\underline{数字の}右から一番目が{\bfseries 偶数の人は右側}の問題を
解いて下さい。
\vspace{-5ex}





\subsection*{問1}

次のC言語の式を評価せよ。
\begin{center}
 \begin{tabular}{|l|c|}\hline
 式 & 評価結果 \\\hline\hline
 \verb;27 / 9/3;& \an{1} \\\hline
 \verb;144/12>=-15+20;& \an{1}  \\\hline
 \verb;!(3>3)*4;& \an{4} \\\hline
 \verb;((0>1)&&(2<4))||((4-2)/2);& \an{1}\\\hline
 \end{tabular}
\end{center}


\subsection*{問2}

次の数式をC言語の式に書き直せ。
ただし$e$は指数とする。

\begin{itemize}

 \item $\ds1+\frac{14+2}{\frac{-2}{8-3}-3}$

       \quad \rule[-3ex]{0pt}{6ex}
       \an{{\ttfamily
       1+(14+2)/(-2/(8-3)-3)
       }}

 \item $2\sin\ds\frac{\theta+\pi}{2}\cos\ds\frac{\theta-\pi}{2}$

       \quad \rule[-3ex]{0pt}{6ex}
       \an{{\ttfamily
       2*sin((theta+pi)/2)*cos((theta-pi)/2)
       }}
       \quad \rule[-3ex]{0pt}{6ex}

 \item $\ds\frac{e(r^n-1)}{r-1}$

       \quad \rule[-3ex]{0pt}{6ex}
       \an{{\ttfamily
       exp(1)*(pow(r,n)-1)/(r-1)
       }}



\end{itemize}





\subsection*{問3}

以下の代入文が上から順番に実行されるとき、全ての代入文の
実行が終わった時点での各変数の値を書け。

\begin{verbatim}
x = 1;
y = (x = 2) + 1;
z = 8 + x;
y = x / 2;
x = x + 4;
y = (x > 1) * 2 + z + (x < 10) * 4;
z = (x + y) * 2;
\end{verbatim}


{\ttfamily x}の値:\an{\quad 6}\\

{\ttfamily y}の値:\an{\quad 16}\\

{\ttfamily z}の値:\an{\quad 44}\\






\vfill




\end{multicols*}


\end{document}
