\documentclass[a4j]{jarticle}
\usepackage{multicol}
\raggedbottom
\addtolength{\textwidth}{4cm}
\addtolength{\textheight}{3.5cm}
\addtolength{\topmargin}{-2.5cm}
\addtolength{\evensidemargin}{-2cm}
\addtolength{\oddsidemargin}{-2cm}
%\setlength{\columnseprule}{.5pt}
\setlength{\columnsep}{4zw}

\makeatletter
\def\verbatim@font{\small\bfseries\ttfamily}
\makeatother



%
% 1か2を指定する。
%
\def\anss{1} % 1 : Question mode 学生に解かせるとき
             % 2 : Answer mode   回答例



\ifnum \anss=1
\def\an#1{\phantom{#1}} % Question mode
\else
\def\an#1{{#1}}  % Answer mode
\fi



\def\ans#1#2{
\ifnum \anss=1
#1
\else
#2
\fi
}




\def\ds{\displaystyle}

\begin{document}
\thispagestyle{empty}

\begin{multicols*}{2}%


\def\subst#1#2{$\ds #1$
 \ $\longrightarrow$\ 
 \underline{\hbox to 5cm{\ttfamily #2}}}



\noindent
\begin{tabular}[t]{|c|cccccccc|}\hline
氏  名 & & & & & & & & \\ \hline
\end{tabular}\\
\begin{tabular}[t]{|c|c|c|c|c|c|c|c|c|c|}\hline
学籍番号 & & & & & & & & \\ \hline
\end{tabular}\\
学籍番号の\underline{数字の}右から一番目が{\bfseries 奇数の人は左側}の問題を
解いて下さい。
\vspace{-5ex}







\subsection*{問1}


{\ttfamily int型変数a,b}を使って、二重ループを用いて
$10\times80$から$99\times89$のかけ算{\ttfamily a*b}を表示するプログラムを書け。

\ifnum \anss=1
\begin{verbatim}
int a = 10, b;

















\end{verbatim}
\else
\begin{verbatim}
  int a = 10, b;
  while(a <= 99){
    b = 80;
    while(b <= 89){
      printf("%d\n", a * b);
      b++;
    }
    a++;
  }
\end{verbatim}
\vspace{2cm}
\fi



\subsection*{問2}

前問の$10\times80$から$99\times89$のかけ算{\ttfamily a*b}のうち、
{\ttfamily aとb}の積が4桁で、かつ
{\ttfamily aと8}の積が2桁で、かつ
{\ttfamily aとb-80}の積が3桁である場合にのみ
{\ttfamily a*b}の結果を表示するプログラムを(二重ループと{\ttfamily if}
を用いて)書け。


\ifnum \anss=1
\begin{verbatim}



























\end{verbatim}
\else
\begin{verbatim}
int a = 10, b;
while(a <= 99){
  b = 80;
  while(b <= 89){
    if(a*b >= 1000 && a*b < 10000){
        if(a*8 >= 10 && a*8 < 100){
          if(a*(b-80) >= 100 && a*(b-80) < 1000){
            printf("  %d\n", a);
            printf("x %d\n", b);
            printf("-------\n");
            printf(" %d\n", a*(b-80));
            printf(" %d\n", a*8);
            printf("-------\n");
            printf("%d\n", a*b);
          }}}
    b++;
  }
  a++;
}

虫食算      正解
  ??          12   
x 8?        x 89   
-------     -------
 ???         108   
 ??          96    
-------     -------
????        1068   
\end{verbatim}
\vfill
\mbox{}
\fi






%以下の配列{\ttfamily a}の、20以上の要素を全て表示したい。
%{\ttfamily for}を使って、そのためのプログラムを書け。
%変数{\ttfamily i}を使ってよい。
%
%
%\ifnum \anss=1
%\begin{verbatim}
%int i, a[5] = {11, 56, 34, 77, 39};
%
%
%
%
%
%\end{verbatim}
%\else
%\begin{verbatim}
%int i, a[5] = {11, 56, 34, 77, 39};
%for(i = 0, i < 5, i++){
%  if(a[i] >= 20){
%    printf("%c ", a[i]);
%  }
%}
%\end{verbatim}
%\fi
%\vspace{5cm}
%
%
%
%\subsection*{問2}
%
%問1のプログラムのフローチャートを描け。
%
%\vspace{10cm}
%
%
%




%\vfill

%\mbox{}

%{\vbox{\vspace{1cm}}}




\noindent
\begin{tabular}[t]{|c|cccccccc|}\hline
氏  名 & & & & & & & & \\ \hline
\end{tabular}\\
\begin{tabular}[t]{|c|c|c|c|c|c|c|c|c|c|}\hline
学籍番号 & & & & & & & & \\ \hline
\end{tabular}\\
学籍番号の\underline{数字の}右から一番目が{\bfseries 偶数の人は右側}の問題を
解いて下さい。
\vspace{-5ex}





\subsection*{問1}




{\ttfamily int型変数a,b}を使って、二重ループを用いて
$100\times80$から$999\times89$のかけ算{\ttfamily a*b}を表示するプログラムを書け。

\ifnum \anss=1
\begin{verbatim}
int a = 100, b;

















\end{verbatim}
\else
\begin{verbatim}
  int a = 100, b;
  while(a <= 999){
    b = 80;
    while(b <= 89){
      printf("%d\n", a * b);
      b++;
    }
    a++;
  }
\end{verbatim}
\vspace{2cm}
\fi



\subsection*{問2}

前問の$100\times80$から$999\times89$のかけ算{\ttfamily a*b}のうち、
{\ttfamily aとb}の積が4桁で、かつ
{\ttfamily aと8}の積が3桁で、かつ
{\ttfamily aとb-80}の積が4桁である場合にのみ
{\ttfamily a*b}の結果を表示するプログラムを(二重ループと{\ttfamily if}
を用いて)書け。



\ifnum \anss=1
\begin{verbatim}





















\end{verbatim}
\else
\begin{verbatim}
int a = 100, b;
while(a <= 999){
  b = 80;
  while(b <= 89){
    if(a*b >= 1000 && a*b < 10000){
      if(a*8 >= 100 && a*8 < 1000){
        if(a*(b-80) >= 1000 && a*(b-80) < 10000){
          printf(" %d\n", a);
          printf("x %d\n", b);
          printf("-------\n");
          printf("%d\n", a*(b-80));
          printf("%d\n", a*8);
          printf("-------\n");
          printf("%d\n", a*b);
        }}}
    b++;
  }
  a++;
}

虫食算      正解
 ???         112  
x 8?        x 89  
-------     -------
????        1008  
???         896   
-------     -------
????        9968   
\end{verbatim}
\fi









%
%以下の配列{\ttfamily b}の、40未満の要素を全て表示したい。
%{\ttfamily for}を使って、そのためのプログラムを書け。
%変数{\ttfamily j}を使ってよい。
%
%
%\ifnum \anss=1
%\begin{verbatim}
%int j, b[5] = {11, 56, 34, 77, 39};
%
%
%
%
%
%\end{verbatim}
%\else
%\begin{verbatim}
%int j, b[5] = {11, 56, 34, 77, 39};
%for(j = 0, j < 5, j++){
%  if(b[j] < 40){
%    printf("%c ", b[j]);
%  }
%}
%\end{verbatim}
%\fi
%\vspace{5cm}
%
%
%
%\subsection*{問2}
%
%問1のプログラムのフローチャートを描け。
%
%\vspace{10cm}
%









%\vfill




\end{multicols*}


\end{document}


%\documentclass[a4j]{jarticle}
%\usepackage{multicol}
%\raggedbottom
%\addtolength{\textwidth}{4cm}
%\addtolength{\textheight}{3.5cm}
%\addtolength{\topmargin}{-2.5cm}
%\addtolength{\evensidemargin}{-2cm}
%\addtolength{\oddsidemargin}{-2cm}
%\setlength{\columnseprule}{.5pt}
%\setlength{\columnsep}{4zw}
%
%\makeatletter
%\def\verbatim@font{\small\bfseries\ttfamily}
%\makeatother
%
%
%\def\an#1{{#1}}
%%\def\an#1{\phantom{#1}}
%
%
%\def\ds{\displaystyle}
%
%\begin{document}
%\thispagestyle{empty}
%
%\begin{multicols*}{2}%
%
%
%\def\subst#1#2{$\ds #1$
% \ $\longrightarrow$\ 
% \underline{\hbox to 5cm{\ttfamily #2}}}
%
%
%
%\noindent
%\begin{tabular}[t]{|c|cccccccc|}\hline
%氏  名 & & & & & & & & \\ \hline
%\end{tabular}\\
%\begin{tabular}[t]{|c|c|c|c|c|c|c|c|c|c|}\hline
%学籍番号 & & & & & & & & \\ \hline
%\end{tabular}\\
%学籍番号の\underline{数字の}右から一番目が{\bfseries 奇数の人は左側}の問題を
%解いて下さい。
%\vspace{-5ex}
%
%
%
%
%
%
%\subsection*{問1}
%
%次のwhileループ(問133参照)をforループに書き直せ。
%\begin{verbatim}
%int i, j;
%i = 31;
%while(i < 58){
%  printf("i = %d\n", i);
%  j = 45;
%  while(j > 12){
%    printf("i = %d, j = %d\n", i, j);
%    j = j - 1;
%  }
%  i = i + 1;
%}
%\end{verbatim}
%
%%\begin{verbatim}
%%  int i, j;
%%  for(i = 31; i < 58; i++){
%%    printf("i = %d\n", i);
%%    for(j = 45; j > 12; j--){
%%      printf("i = %d, j = %d\n", i, j);
%%    }
%%  }
%%\end{verbatim}
%\begin{verbatim}
%
%
%
%
%
%
%
%\end{verbatim}
%
%
%
%
%\subsection*{問2}
%
%
%200から3500までの整数のうち、
%7で割ると2あまり、23で割ると10余る
%数を全て表示するプログラム
%のフローチャートを描け。
%
%
%\vspace{10cm}
%
%
%\vfill
%
%\mbox{}
%
%%{\vbox{\vspace{1cm}}}
%
%
%
%
%\noindent
%\begin{tabular}[t]{|c|cccccccc|}\hline
%氏  名 & & & & & & & & \\ \hline
%\end{tabular}\\
%\begin{tabular}[t]{|c|c|c|c|c|c|c|c|c|c|}\hline
%学籍番号 & & & & & & & & \\ \hline
%\end{tabular}\\
%学籍番号の\underline{数字の}右から一番目が{\bfseries 偶数の人は右側}の問題を
%解いて下さい。
%\vspace{-5ex}
%
%
%
%
%\subsection*{問1}
%
%次のwhileループ(問133参照)をforループに書き直せ。
%\begin{verbatim}
%int i, j;
%i = 58;
%while(i > 31){
%  printf("i = %d\n", i);
%  j = 12;
%  while(j < 54){
%    printf("i = %d, j = %d\n", i, j);
%    j = j + 1;
%  }
%  i = i - 1;
%}
%\end{verbatim}
%
%%\begin{verbatim}
%%  int i, j;
%%  for(i = 58; i > 31; i--){
%%    printf("i = %d\n", i);
%%    for(j = 12; j < 54; j++){
%%      printf("i = %d, j = %d\n", i, j);
%%    }
%%  }
%%\end{verbatim}
%\begin{verbatim}
%
%
%
%
%
%
%
%\end{verbatim}
%
%
%
%
%\subsection*{問2}
%
%200から1500までの整数のうち、
%11で割り切れて、17で割ると1余る数を全て表示するプログラム
%のフローチャートを描け。
%
%
%
%
%
%%\vfill
%
%
%
%
%\end{multicols*}
%
%
%\end{document}
