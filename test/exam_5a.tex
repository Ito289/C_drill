\documentclass[a4j]{jarticle}
\usepackage{multicol}
\usepackage{fancybox}

\raggedbottom
\addtolength{\textwidth}{4cm}
\addtolength{\textheight}{4cm}
\addtolength{\topmargin}{-2.5cm}
%\addtolength{\topmargin}{-1.5cm}
\addtolength{\evensidemargin}{-2cm}
\addtolength{\oddsidemargin}{-2cm}
\setlength{\columnseprule}{.5pt}
\setlength{\columnsep}{4zw}

\makeatletter
\def\verbatim@font{%\small
\bfseries\ttfamily}
\makeatother


%
% 1か2を指定する。
%
\def\anss{1} % 1 : Question mode 学生に解かせるとき
             % 2 : Answer mode   回答例



\ifnum \anss=1
\def\an#1{\phantom{#1}} % Question mode
\else
\def\an#1{{#1}}  % Answer mode
\fi


\def\ans#1#2{
\ifnum \anss=1
#1
\else
#2
\fi
}

\newcounter{toi}
\setcounter{toi}{1}
\def\toi{%
%\subsubsection*{問\thetoi}
\bigskip\bigskip\noindent
\shadowbox{\bfseries\large 問題\thetoi}
%\addcontentsline{toc}{subsubsection}{問\thetoi}
\addtocounter{toi}{1}
%\nopagebreak[4]\bigskip\nopagebreak[4]
}





\def\ds{\displaystyle}

\begin{document}
\thispagestyle{empty}
\pagestyle{empty}

\begin{multicols*}{2}%


\def\subst#1#2{$\ds #1$
 \ $\longrightarrow$\ 
 \underline{\hbox to 5cm{\ttfamily #2}}}



\noindent
\begin{tabular}[t]{|c|cccccccc|}\hline
氏  名 & & & & & & & & \\ \hline
\end{tabular}\\
\begin{tabular}[t]{|c|c|c|c|c|c|c|c|c|c|}\hline
学籍番号 & & & & & & & & \\ \hline
\end{tabular}\\



\toi
次のC言語の式を評価せよ。
\begin{itemize}
 \item \verb@77*4/8+1@  解答:\ans{_____}{39}
 \item \verb@(-4<1)&&(9>2)@  解答:\ans{_____}{1}
 \item \verb@5<4<3@  解答:\ans{_____}{1}
\end{itemize}


\toi
次の数学の式をC言語の式に書き直せ。
\begin{itemize}
 \item $-2+\ds\frac{1-\ds\frac{3+1}{4}}{3-1}$\\[2ex]
       解答:\ans{}{{\ttfamily\small
       -2+(1-(3+1)/4)/(3-1)}}\\
 \item $\sqrt{2x^2+2xy+y^2+1}+\sin\ds\frac{x}{2}$\\[2ex]
       解答:\ans{}{{\ttfamily\small
       sqrt(2*x*x+2*x*y+y*y+1)+sin(x/2)}}
\end{itemize}


\toi
次の数式を代入文で書こうとしたが、間違っている。
訂正せよ。
\begin{itemize}
 \itemsep=5ex
 \item $x=\ds\frac{-\frac{h}{i+2}-j}{k+l}$
      $\quad\longrightarrow\quad$
      \ans{\ttfamily x=[-h\{i+2\}-j]/\{k+l\};}
      {\ttfamily x=(-h/(i+2)-j)/(k+l);}
% \item $h=\ds\frac{1}{3}+e^{2x}$
%      $\quad\longrightarrow\quad$
%      \ans{\ttfamily h=1./3+pow(e,2x);}
%      {\ttfamily h=1./3+pow(e,2*x);
%      または h=1./3+exp(2*x);}
\end{itemize}




%\def\nums{\hfill 1 2 3 4 5}
\def\nums{\hfill 1&2&3&4&5}

\toi
以下の各分野の「現在の」理解度について、
当てはまる数字を選べ。
 {\bfseries\footnotesize 1.理解している/
2.だいたい分かっている/
3.まあなんとなく/
4.まだよく分かっていない/
5.さっぱり}
\begin{center}\small
% \begin{tabular}{|r|c|}\hline
 \begin{tabular}{|r||c|c|c|c|c|c|}\hline
 式の評価&\nums\\\hline
 評価順序&\nums\\\hline
 変数への代入&\nums\\\hline
 整数型と実数型の違い&\nums\\\hline
 {\ttfamily printf}関数&\nums\\\hline
 配列&\nums\\\hline
 {\ttfamily while}&\nums\\\hline
 {\ttfamily for}&\nums\\\hline
 二重ループ&\nums\\\hline
 無限ループ&\nums\\\hline
 剰余&\nums\\\hline
 {\ttfamily if}&\nums\\\hline
 {\ttfamily else}&\nums\\\hline
 {\ttfamily if}とループの組合せ&\nums\\\hline
 {\ttfamily \&\&}や{\ttfamily ||}&\nums\\\hline
 フローチャートを読む&\nums\\\hline
 フローチャートを描く&\nums\\\hline
 ポインタ&\nums\\\hline
 \end{tabular}
\end{center}





\toi
計算せよ。\\
$\ds\sum_{m=1}^{5}\left(2m+1\right)=$
\ans{_____}{35}






\toi
以下の代入文が上から順に実行されたとき、
各変数の値を書け。
\begin{verbatim}
    int x, y, z;
    x = 2;
    y = 4;
    x = x + 1;
    y++;
    x -= y;
    z = x + y;
\end{verbatim}
\noindent
解答\\
xの値:\ans{___}{-2} %
yの値:\ans{___}{5} %
zの値:\ans{___}{3}



\toi
符号無しlong型の変数{\ttfamily a}と、
文字型の変数{\ttfamily c}
の変数宣言を書け。

\ifnum \anss=1
\begin{verbatim}



\end{verbatim}
\else
\begin{verbatim}
    unsigned long int a;

    char c;
\end{verbatim}
\fi






\toi
何が表示されるか。
\begin{verbatim}
    float a=4.8, b=2.4, x;
    int m=3, n=2;
    x = a / b + n / m;
    if(x > 2){
      printf("2より大\n");
    }else{
      printf("2より小\n");
    }
\end{verbatim}
\noindent
解答:\ans{}{2より小}


\toi
何が表示されるか。
\begin{verbatim}
    int c[4] = {59, 12, 45, 60};
    printf("%d %d %d\n", c[0]-1, c[1-1], c[3]);
\end{verbatim}

\noindent
解答:\ans{}{{\ttfamily 58 59 60}}




\toi
for文を使って、配列{\ttfamily Value[256]}
の全ての要素の和を計算して
表示したい。以下のプログラムの
下線部を埋めよ。ただし{\ttfamily Value}はすでに
宣言され初期化されているとする。

\ifnum \anss=1
\begin{verbatim}
int k, add=0;
for( ____________________ ){
   add += Value[k];
}
printf("%d\n", add);
\end{verbatim}
\else
\begin{verbatim}
int k, add=0;
for(k = 0; k < 256; k++){
   add += Value[k];
}
printf("%d\n", add);
\end{verbatim}
\fi


\toi
何が表示されるか。
\begin{verbatim}
int x, y, n = 0;
for(x = 0; x < 10; x++){
  for(y = 0; y < 10; y++){
    n++;
  }
}
printf("n is %d\n", n);
\end{verbatim}
解答:\ans{__________}{{\ttfamily n is 100}}





\toi
配列{\ttfamily a}の要素のうち、
{\ttfamily x}より大きいものを全て表示するプログラムの続きを、
ループを使って書け。
ただし{\ttfamily x}はすでに宣言され初期化されているとする。
\ifnum \anss=1
\begin{verbatim}
int i, a[5] = {2, 44, 99, 35, 41};














\end{verbatim}
\else
\begin{verbatim}
int i, a[5] = {2, 44, 99, 35, 41}, x = 40;
i = 0;
while(i < 5){
  if(a[i] > x){
  printf("%d ", a[i]);
  }
  i++;
}
または
int i, a[5] = {2, 44, 99, 35, 41}, x = 40;
for(i = 0; i < 5; i++){
  if(a[i] > x){
  printf("%d ", a[i]);
  }
}
\end{verbatim}
\fi


\toi
以下の代入文が上から順に実行されたとき、
各変数の値を書け。
\begin{verbatim}
int x = 2, *p, y = 4;
p = &x;
x = 3;
y = *p + 1;
\end{verbatim}
\noindent
解答\\
{\ttfamily  x}の値:\ans{___}{3} %
{\ttfamily *p}の値:\ans{___}{3} %
{\ttfamily  y}の値:\ans{___}{4}


\toi
次のプログラムに対応するフローチャートを描け。
\begin{verbatim}
int i, n = 0;
for(i = 7; i > -2; i--){
  printf("%d ", n);
  n += 2 * i + 1;
}  
\end{verbatim}


\vfill\vfill\vfill\vfill\vfill\vfill\vfill\vfill



\end{multicols*}


\end{document}
