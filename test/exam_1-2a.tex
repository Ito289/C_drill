\documentclass[a4j]{jarticle}
\usepackage{multicol}
\raggedbottom
\addtolength{\textwidth}{4cm}
\addtolength{\textheight}{3.5cm}
\addtolength{\topmargin}{-2.5cm}
\addtolength{\evensidemargin}{-2cm}
\addtolength{\oddsidemargin}{-2cm}
\setlength{\columnseprule}{.5pt}
\setlength{\columnsep}{4zw}

\makeatletter
\def\verbatim@font{\small\bfseries\ttfamily}
\makeatother


%
% 1か2を指定する。
%
\def\anss{1} % 1 : Question mode 学生に解かせるとき
             % 2 : Answer mode   回答例



\ifnum \anss=1
\def\an#1{\phantom{#1}} % Question mode
\else
\def\an#1{{#1}}  % Answer mode
\fi


\def\ans#1#2{
\ifnum \anss=1
#1
\else
#2
\fi
}


\def\ds{\displaystyle}

\begin{document}
\thispagestyle{empty}

\begin{multicols*}{2}%


\def\subst#1#2{$\ds #1$
 \ $\longrightarrow$\ 
 \underline{\hbox to 5cm{\ttfamily #2}}}



\noindent
\begin{tabular}[t]{|c|cccccccc|}\hline
氏  名 & & & & & & & & \\ \hline
\end{tabular}\\
\begin{tabular}[t]{|c|c|c|c|c|c|c|c|c|c|}\hline
学籍番号 & & & & & & & & \\ \hline
\end{tabular}\\
学籍番号の\underline{数字の}右から一番目が{\bfseries 奇数の人は左側}の問題を
解いて下さい。
\vspace{-5ex}


\subsection*{問1}

以下の代入文が上から順番に実行されるとき、全ての代入文の
実行が終わった時点での各変数の値を書け。

\begin{verbatim}
y = 1;
x = 2;
z = sqrt(9);
x = x + 14;
y = x / 2;
x = x + 4;
y = z = z+x * 4;
z = (z - 3) / 40*2;
\end{verbatim}


{\ttfamily x}の値:\an{\quad 20}\\

{\ttfamily y}の値:\an{\quad 83}\\

{\ttfamily z}の値:\an{\quad 4}\\


\subsection*{問2}

%代入文「{\ttfamily x = 3}」と「{\ttfamily y = 5}」
%によって変数{\ttfamily x, y}に値が代入されているとき、
すでに変数{\ttfamily x, y}に何かの値が代入されているとき、
変数{\ttfamily x, y}の値を入れ換えるための
代入文の組を書け。

変数{\ttfamily x, y}以外に
変数{\ttfamily z}を使ってよい。

\ifnum \anss=1
\vspace*{3cm}
\else
\begin{verbatim}
z = x;
x = y;
y = z;
\end{verbatim}\vspace*{1cm}
\fi



\subsection*{問3}
次の数学の数式をC言語の式に書き直せ。\\

$2\sigma+\frac{\mu}{2}$
\an{\quad {\ttfamily 2*sigma+mu/2}}

$\frac{\sqrt{2}+y}{\frac{2}{x-5}+z}$
\an{\quad {\ttfamily (sqrt(2)+y)/(2/(x-5)+z)}}

$x^2+(x+1)^{10}$
\an{\quad {\ttfamily x*x+pow(x+1,10)}}



\subsection*{問4}


1. 実数型の変数{\ttfamily x1}と{\ttfamily y2}の変数宣言を書け。\\
\an{{\ttfamily double x1, y2;} または {\ttfamily float x1, y2;}}\\

2. {\ttfamily 10/3}の評価結果は  \an{3}  である。




%当てはまるものにチェックを付けよ。
%
%\begin{itemize}
% \itemsep=-3pt
% \item 「代入」は: □よく分からない。□理解した。
% \item 評価順序は: □よく分からない。□理解した。
% \item 変数の値の入れ換えは: □できない。□できる。
%\end{itemize}



%\vspace{2cm}

\vfill

%\mbox{}



%{\vbox{\vspace{1cm}}}




\noindent
\begin{tabular}[t]{|c|cccccccc|}\hline
氏  名 & & & & & & & & \\ \hline
\end{tabular}\\
\begin{tabular}[t]{|c|c|c|c|c|c|c|c|c|c|}\hline
学籍番号 & & & & & & & & \\ \hline
\end{tabular}\\
学籍番号の\underline{数字の}右から一番目が{\bfseries 偶数の人は右側}の問題を
解いて下さい。
\vspace{-5ex}




\subsection*{問1}

%代入文「{\ttfamily a = 1.23}」と「{\ttfamily b = 5.67}」
%によって変数{\ttfamily a, b}に値が代入されているとき、
すでに変数{\ttfamily a, b}に何かの値が代入されているとき、
変数{\ttfamily a, b}の値を入れ換えるための
代入文の組を書け。

変数{\ttfamily a, b}以外に
変数{\ttfamily c}を使ってよい。

\ifnum \anss=1
\vspace*{3cm}
\else
\begin{verbatim}
c = a;
a = b;
b = c;
\end{verbatim}\vspace*{1cm}
\fi



\subsection*{問2}

以下の代入文が上から順番に実行されるとき、全ての代入文の
実行が終わった時点での各変数の値を書け。

\begin{verbatim}
z = 3;
y = 1;
x = sqrt(4);
z = z + x + 14;
y = y + x / 2;
x = x + 4;
y = z + x * y;
y = (y - 3) / 14*2;
\end{verbatim}


{\ttfamily x}の値:\an{\quad 6}\\

{\ttfamily y}の値:\an{\quad 4}\\

{\ttfamily z}の値:\an{\quad 19}\\




\subsection*{問3}
次の数学の数式をC言語の式に書き直せ。\\

$2\theta+\frac{\pi}{2}$
\an{\quad {\ttfamily 2*theta+pi/2}}

$\frac{-z+\frac{x-5}{2}}{2+y}$
\an{\quad {\ttfamily (-z+(x-5)/2)/(2+y)}}

$\sqrt{x}+a^p$
\an{\quad {\ttfamily sqrt(x)+pow(a,p)}}


\subsection*{問4}



1. 符合無し整数(ただしlong)型
の変数{\ttfamily aとbとc}の変数宣言を書け。\\
\an{{\ttfamily unsigned long int a, b, c;}}\\


2. {\ttfamily 3/10}の評価結果は  \an{0}  である。




%
%当てはまるものにチェックを付けよ。
%
%\begin{itemize}
% \itemsep=-3pt
% \item 「代入」は: □理解した。□よく分からない。
% \item 評価順序は: □理解した。□よく分からない。
% \item 変数の値の入れ換えは: □できる。□できない。
%\end{itemize}

%\vspace{2cm}

%\vfill












\end{multicols*}


\end{document}
